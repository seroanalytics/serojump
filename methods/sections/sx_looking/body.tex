\section{Looking forward}
\paragraph{}n summary, we have shown the ability of \textbf{Algorithm~\ref{alg:rjmcmc_C}} to recover the state varibles $\{\theta\}$ from simulated serological data. Additionally, we have shown how well the correlation of protection and antibody kinetics functions are recovered, showing that they are well-recovered for all six of our models chosen. The model cannot infer individual-level exposure for those not infected. However, this is unsurprising as the kinetics between these groups are equivalent. The proportion of the population exposed is well recovered, however. At high levels of variability in the antibody kinetics, the ability of the model to infer the exposure time on the individual level weakens and the epidemic curves starts to differ from the simulated curve. Finally, the COP infection and infection status are well-recovered for all models considered, suggesting that their inference is more reliable in the face of high-level individual-level variability in antibody kinetics. 

\paragraph{}RJMCMC algorithms have been used in infectious disease modelling previously. Hendrick? 

\paragraph{}These models are incredibly useful for several reasons: Immunobridging?

\paragraph{}Extensions in the future
\begin{itemize}
\item Add the possibility of inferring multiple exposures for an individual
\item Add hierarchical effects to antibodies kinetics and correlates of protection
\item Add inferring for multiple biomarkers and antigenically varied pathogens to improve inference
\item Methods development: parallel tempering? Accessibility of method to others 
\end{itemize}


This document has provided details of the theoretical underpinning and implementation of a reversible jump mcmc algorithm, which can infer important epidemiology and immunological information from individual-level serological data. On the individual level, it can infer the exposure status, infection status, infection timines, the antibody kinetics and the correlate of protection for each individual.
\paragraph{}To conclude, this documents provides a walkthrough of how to implement a reversible jump algorithm to infer serological data. We hope this technique will be useful for inferring epidemiological information in a pathogen-agnostic setting, particularly pathogens where intense surveillance is challenging. We also hope this document sheds light on a mathematically complex but powerful inferring tool and encourages others to implement similar algorithms in other areas of health science which require the exploration of multidimensional model spaces. 
