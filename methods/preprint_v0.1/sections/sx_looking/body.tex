\section{Looking forward}
\paragraph{}We have shown that \textbf{Algorithm~\ref{alg:rjmcmc_B}} can recover the state variables $\{\theta, \mathbf{E}, \mathbf{E^{\tau}}, \mathbf{Z}\}$, the correlate of protection and the antibody kinetics from simulated serological data.\cite{Menezes2023-ti} The correlate of protection and infection states are well-recovered across all datasets simulated for contrasting COP curves. As the variability of the antibody kinetics increases, the ability of the algorithm to infer the exposure time on the individual level weakens, but, on the population level, the epidemic curve is still accurately recovered.  

\paragraph{}In the current framework, the user can choose the functional form of the antibody kinetics and COP, allowing for flexibility in the inference methods, which can be tailored to the pathogen being analysed. For example, if the timing of a vaccination programme is known (and stored in a vector, $\mathbf{V}$), then vaccination kinetics could be added into the antibody kinetics function and inferred within the framework. This is useful for prospective cohort studies which follow vaccinated cohorts and infer correlates of protection/infection by looking at breakthrough infections. Additionally, hierarchical effects can be added to either the antibody kinetics or the correlates of protection such that the impact of host factors on both of these immunological processes can be evaluated. This is crucial for determining the impact of host factors such as age and exposure history on resulting immunological characteristics over the season. Hierarchical models like this can be used to assess the legitimacy of immune imprinting or original antigenic sin by assessing how exposure history alters the COP. 

\paragraph{}Future extensions to this framework include i) adding the possibility of inferring multiple exposures for an individual over an epidemic period and ii) adding inference for multiple biomarkers and antigenically varied pathogens. Inferring multiple exposures per individual over a larger timeframe will allow the exploration of the impact of poorly understood longer-term immunological phenomena such as original antigenic sin, immune imprinting, etc. Adding inference for multiple biomarkers can help better infer infection status for antigenically varied pathogens such as influenza, SARS-CoV-2, etc. Combining these two new features permits the life history of infections to be inferred given immunological titre landscapes, similar to the methods such as the serosolver framework\cite{Hay2020-pr}. However, these extensions require a high number of parameters, greatly increasing the inferred state space and leading to sampling times which may be prohibitively long. These could be overcome by optimising existing analysis, e.g. i) coding the likelihood in C++, or ii) finding more optimal scalings in the RJ-MCMC to ensure optimal convergence. Alternatively, adding a population-level MCMC algorithm (such as parallel tempering) with a reversible jump could improve mixing considerably, allowing for more complex frameworks to be evaluated.

%\paragraph{}RJ-MCMC have been used previously for serological inference.\cite{} However, these inference algorithm differ from the methods outlined here as the infer infection only and also infer force of infection rates across multiple seasons. By only infering infection these methods re able to estab These merhin that the infection times given a hazard ratio function. In these frameworks the attack rate for each pathogen can be recovered , however these inference methods do no recover exposure rates, making correlate of protection unrecoverable. Further exrrensions to this framework woul dbe to incorporate atttack rates into the prior disribution for the timing of infection, linking these methods into one framework. 


\paragraph{}This document has provided details of the theoretical underpinning and implementation of an RJ-MCMC algorithm, which can infer important epidemiology and immunological information from individual-level serological data. Given broad structural forms for antibody kinetics and the correlation of protection, it can recover the exposure status, exposure times, and infection status. We hope this technique will be useful for inferring epidemiological information in a pathogen-agnostic setting, particularly pathogens for which intense surveillance is challenging. We also hope this document sheds light on a mathematically complex but powerful inference tool and encourages others to implement similar algorithms in other health science areas requiring the exploration of multidimensional model spaces. 
